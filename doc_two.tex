% ********************************* HEADERS ***********************************
\documentclass{article}
% setcounter{secnumdepth}{5}
\usepackage[top=.75in, bottom=.75in, left=.50in,right=.50in]{geometry}
\usepackage{fancyhdr}
\usepackage{titling}
\pagestyle{fancy}
\lhead{EECS 281 - Data Structures and Algorithms}
\rhead{\thepage}
\usepackage{color}
\usepackage{graphicx}
\newcommand{\heart}{\ensuremath\heartsuit}
\usepackage{listings}
\lstset{
  % language=C++,
  showstringspaces=false,
  basicstyle={\small\ttfamily},
  numberstyle=\tiny\color{gray},
  keywordstyle=\color{blue},
  commentstyle=\color{dkgreen},
  stringstyle=\color{dkgreen},
}
\usepackage[colorlinks,urlcolor={blue}]{hyperref}
%\setlength{\parskip}{1em}
\usepackage{parskip}
\usepackage{indentfirst}
\setlength{\droptitle}{-4em}
% \usepackage{concmath}
% \usepackage[T1]{fontenc}
\renewcommand{\theenumi}{\Alph{enumi}}
% ********************************* HEADERS ***********************************
\pagenumbering{gobble}
\begin{document}
\title{\textbf{EECS 281 Lab 8 - Fragmented Data Structures (part 2)}}
\author{Due Tuesday, March 15th, 2017 at 11:59pm}
\date{}
\maketitle
\fancypagestyle{firststyle}
{
   \fancyfoot[L]{Written by Aaryaman Sagar for EECS 281 @ The University of
                 Michigan - Ann Arbor}
   \pagestyle{empty}
}
\thispagestyle{firststyle}

\section{An untyped memory allocator}

Since you now have had sufficient experience allocating memory on the heap via
\texttt{new}, in this lab you will be using the transparent linked list from
last lab to finish writing a memory allocator to allocate memory on the heap!

\subsection{Concepts}
The memory allocator you will write will have an interface very similar to the
C standard library function \texttt{malloc()} to allocate memory on the heap.

Allocators including the C++ \texttt{new} operator get memory on the heap
through function calls to the operating system.  The OS finds new memory
blocks that can be used by the process and returns a pointer to the start of
the usable memory block.  Your allocator will track the memory available from
all previous requests to the operating system in a fragmented linked list.

A fragmented data structure is one that comprises of elements stored
discontinuously and possibly of different sizes.  The below two pictures
illustrate the difference between fragmented and non-fragmented data
structures.

\begin{figure}[!htb]
\centering
\includegraphics[height=1.7cm]{fragmented_memory}
\caption{Fragemented memory}
\end{figure}

\begin{figure}[!htb]
\centering
\includegraphics[height=1.7cm]{non_fragmented_memory}
\caption{Non fragemented memory}
\end{figure}

A memory request is served by returning a block of memory to the user as a
\texttt{void*} pointer.  This allocated memory is large enough to store the
amount of data the user has asked for.  The allocator keeps track of the
memory segments fetched from all previous calls to the operating system and
how large each segment is by storing a header with the size information just
before the actual usable memory.  The memory map looks like this, with the
black rectangles denoting the headers and the grey the actual memory blocks
that can be used by the user.

\begin{figure}[!htb]
\centering
\includegraphics[height=1.7cm]{memory_storage}
\caption{Memory storage}
\end{figure}

In the above image, the pointer that will be returned to the user for usage
will start at the starting location of the grey area, and the header right
before that will contain the length of that grey area.  This way the allocator
knows how much memory each allocated block points to, so that if the
\texttt{free()} function is called on that memory, it can be put back into the
linked list of headers.

The linked list of headers will be sorted in order of the addresses of the
headers, so if a header has an address that is smaller than the address of
another it appears earlier on in the linked list as compared to the other.
The linked list of headers will resemble the list shown in figure 4 below,
with each header pointing to the next header in the logically increasing order
or addresses (for example here the address of the first header on the left is
smaller than the address for the middle header which is smaller than the
address for the last header)

\begin{figure}[!htb]
\centering
\includegraphics[height=2.3cm]{linked_list_headers}
\caption{Linked list headers}
\end{figure}

When the user asks for memory, the block with the right amount of memory will
be taken out of the linked list and returned to the user, so for example if
the user asked for a large portion of memory then the allocator can
return a portion of the second segment to the user.  Subsequently the free
list will be updated to look like so, with the user using the memory in red.
The header right before the red portion of memory will contain the length of
the memory in the red portion, so that if that memory is freed, the allocator
knows the amount of memory that can be stored in that segment.

\begin{figure}[!htb]
\centering
\includegraphics[height=2.3cm]{after_allocation}
\caption{After an allocation}
\end{figure}

When a user program wants to deallocate the same memory (like a call to
delete) they have to pass the pointer that was allocated via the
\texttt{malloc()} call to \texttt{free()}.  The free function is supposed to
backtrack from the pointer the user passed (which will be pointing to the
beginning of the allocated segment, i.e.  the grey area) back to the header
and put the header back into the linked list.  This can be done via a simple
pointer cast to the header type and decrement by one.  Analagously if you
wanted to get the location of an object of type \texttt{int} right before a
\texttt{void*} pointer \texttt{ptr}, you would do

\begin{lstlisting}
int* pointer_right_before = static_cast<int*>(ptr) - 1;
\end{lstlisting}

The free function will coalesce memory blocks if they are right by each other
into one bigger block to save on memory used to store headers.  This can be
done through a call to \texttt{coalesce()}.  If the two headers can be
coalesced then the function will coalesce them and return a pointer to the
coalesced header, the function returns a \texttt{nullptr} to denote failure.

\begin{figure}[!htb]
\centering
\includegraphics[height=2.3cm]{before_coalescing}
\caption{After a deallocation (before coalescing)}
\end{figure}

\begin{figure}[!htb]
\centering
\includegraphics[height=2.3cm]{after_deallocation}
\caption{After a deallocation (after coalescing)}
\end{figure}

\newpage
\subsection{Implementation}
You will be given three modules, one containing the code and declarations for
the allocator, one for the interface to interact with the operating system and
the last one will be your linked list implemenatation from the previous lab
You will fill in code in the \texttt{eecs281malloc.cpp} file in the functions
with the \texttt{TODO} symbols.

\subsubsection{\texttt{malloc()}}
All pointers returned via an untyped allocator have to be aligned to the
maximum alignment on the system (the concept of alignment is beyond the scope
of this course).  Therefore, the first thing \texttt{malloc()} should do is to
align the \texttt{amount} variable to the max alignment on the system via a
call \texttt{round\_up\_to\_max\_alignment()}.

Then the \texttt{malloc()} function should find the first block with a memory
segment that is large enough to serve the user's request.  This is called a
"first fit algorithm".  If there is no block with a memory segment large
enough to serve the user's request, \texttt{malloc()} will call
\texttt{extend\_heap(int)} with a parameter that is a sum of the aligned
amount variable and the size of a header.  You can get the sizeof a header by
calling \texttt{sizeof(Header\_t)}.  This returns a pointer to memory that can
be longer in length than the memory you had asked for.  After calling
\texttt{extend\_heap()}, you need to construct a header in the location
returned by the call to \texttt{extend\_heap()} by calling
\texttt{make\_header()}.  This header then needs to be inserted into the free
list while maintaining its sorted invariant.  At this point the free list will
have a header that is big enough to serve the user's request.  Insert that
header into the linked list after taking out the required memory via a call to
\texttt{remove\_memory()} and then return the pointer to the user.

It might be useful to use the \texttt{std::find\_if()} algorithm with a custom
functor or lambda function that captures the value of the amount that the user
has asked for.  Another function that might be useful to insert a node into
the linked list in a sorted manner is \texttt{insert\_sorted()}.  It's
implementation has been left empty for you to fill and use.

\subsubsection{\texttt{free()}}
The user program must call free for every pointer that it has allocated with
\texttt{malloc()} before terminating the program to avoid memory leaks.  The
\texttt{free()} function puts the memory back in the free list to avoid
repeated calls to the operating system.

The first thing to be done is to fetch a pointer to the header object right
before the pointer passed to \texttt{free()} by the user.  If the user passes
a pointer to memory that has not been returned by a previous call to
\texttt{malloc()} then the behavior of the \texttt{free()} function is
undefined.  To get a pointer to the header right before the memory allocated
by new one would need to cast the address to a pointer type and then subtract
one from it

\begin{lstlisting}
Header_t* header = static_cast<Header_t*>(address) - 1;
\end{lstlisting}

A function called \texttt{coalesce()} should be provided to you declared and
defined in an anonymous namespace.  This function coalesces two headers into
one if they are right after each other in the memory map with nothing in
between.  If the coalescing is not possible, then this function returns a
nullptr and does not modify the contents in either header.  After every call
to \texttt{free()} you have to ensure that all blocks that are coalscable are
coalesced.

\subsection{Implementation API}
While writing a solution to this lab you will be using a lot of functions that
are already written for you.  Here is a description of some of those
functions, mirrored in the comments right above their public declarations.

\subsubsection{\texttt{extend\_heap(int)}}
The \texttt{extend\_heap(int)} function requests memory from the OS that is at
least as big as the amount passed to the function.  It then returns this
uninitialized portion of memory back to the user as a pair
(\texttt{std::pair<void*, int>}) containing the pointer to the memory and an
integer with the actual size of the memory block.  Keep in mind that the
amount of memory allocated might be larger than the amount that you had
requested, your allocator code should take this into consideration!

\subsubsection{\texttt{round\_up\_to\_max\_alignment(int)}}
This function accepts an integer and rounds it up to the maximum alignment on
the system, why this is needed is beyond the scope of the course.  This
function should be called on the amount variable passed to \texttt{malloc()}
before anything else.

\subsubsection{\texttt{make\_header(void*, int)}}
The \texttt{make\_header()} function accepts a \texttt{void*} pointer that
points to allocated memory (allocated from previous calls to the OS) and an
integer that contains the length of the allocated memory.
\texttt{make\_header()} then constructs a header object in the memory location
specified and returns a \texttt{Header\_t*} pointer to the header.  If there
is an error and the header cannot be constructed in the amount of memory
given, the function returns a \texttt{nullptr}

\subsubsection{\texttt{remove\_memory(Header\_t*, int)}}
\texttt{remove\_memory()} accepts a header pointer and an integer that denotes
the amount of memory that is to be removed from the memory segment following
the header and returns a pointer to a header for whatever memory was left
after removing the given amount of memory.  If there was just enough memory
following the header to serve the request, then \texttt{remove\_memory()}
returns the same header pointer that was passed to the function.  If there is
more than sufficient memory then the function returns a header pointer that is
different from the passed pointer.  In which case, your allocator is supposed
to insert that header back into the sorted linked list.

\subsubsection{\texttt{coalesce(Header\_t*, Header\_t*)}}
This function accepts two header pointers and coalesces them into one header
if they are adjacent (after taking into consideration the memory segments
right after the headers).  If the two headers cannot be coalesced into one
then this function returns a nullptr to denote failure.  The contents of the
headers are left unchanged in the event of a failure.


\subsection{Appendix}

\subsubsection{\texttt{static} and anonymous namespaces}
Free functions that are private to an implementation (\texttt{.cpp}) file file
are generally either put in an anonymous namespace or declared to be static,
both achieve the same effect.
\begin{lstlisting}
static void helper_function() { ... }
namespace {
    void helper_function() { ... }
}
\end{lstlisting}

And the C++ standard favors neither as a superior alternative, this lab uses
the anonymous namespace pattern and as such all "helper" functions are put in
an anonymous namespace.  Therefore when you define helper functions put their
declarations and definitions in an anonymous namespace.  There are several
examples of how to do this in the \texttt{eecs281malloc.cpp} file.

\subsubsection{Algorithms in the \texttt{<algorithm>} header}
The \texttt{<algorithm>} header of the C++ standard library includes several
functions that generalize algorithms on a range when given two iterators.  You
might find it useful to use a couple of functions from this header.

\subsubsection{Pointer arithmetic and \texttt{void*} pointers}
If given a typed pointer to a location in memory, subtracting 1 from the
pointer will give you a pointer that points to the previous object of the same
type if the elements were stored in an array.  You will likely find pointer
arithmetic useful when you are given a pointer to a user segment and you want
to get a pointer to the header right before that pointer.

\texttt{void*} pointers denote untyped addresses in memory.  They can be used
to track starting addresses, ending addresses and so on.  They just point to
memory and cannot be used for anything other than casting to and from other
pointer types and integer types.

If you want to use a \texttt{void*} pointer, you must cast them to pointers of
another type by using a \texttt{static\_cast<>}.  For example
\begin{lstlisting}
Header_t* header = static_cast<Header_t*>(address) - 1;
\end{lstlisting}

\subsubsection{How typed memory allocators like \texttt{operator new} work}
Typed memory allocators work surprisingly similarly to the untyped memory
allocator that you will implement in this lab.  The only difference is that
typed memory allocators will construct an object in the location in memory
that is returned to the user prior to their call terminating.  In fact on
several C++ implemenations \texttt{operator new} just calls \texttt{malloc()}
and then constructs an object into the memory location returned by
\texttt{malloc()}

\end{document}


