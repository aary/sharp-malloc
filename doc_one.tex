% ********************************* HEADERS ***********************************
\documentclass{article}
\usepackage[top=.75in, bottom=.75in, left=.50in,right=.50in]{geometry}
\usepackage{fancyhdr}
\usepackage{titling}
\pagestyle{fancy}
\lhead{EECS 281 - Data Structures and Algorithms}
\rhead{\thepage}
\usepackage{color}
\usepackage{graphicx}
\newcommand{\heart}{\ensuremath\heartsuit}
\usepackage{listings}
\lstset{
  % language=C++,
  showstringspaces=false,
  basicstyle={\small\ttfamily},
  numberstyle=\tiny\color{gray},
  keywordstyle=\color{blue},
  commentstyle=\color{dkgreen},
  stringstyle=\color{dkgreen},
}
\usepackage[colorlinks,urlcolor={blue}]{hyperref}
%\setlength{\parskip}{1em}
\usepackage{parskip}
\usepackage{indentfirst}
\setlength{\droptitle}{-4em}
% \usepackage{concmath}
% \usepackage[T1]{fontenc}
\renewcommand{\theenumi}{\Alph{enumi}}
% ********************************* HEADERS ***********************************
\pagenumbering{gobble}
\begin{document}
\title{\textbf{EECS 281 Lab 8 - Fragmented Data Structures}}
\author{Due Tuesday, March 15th, 2017 at 11:59pm}
\date{}
\maketitle
\fancypagestyle{firststyle}
{
   \fancyfoot[L]{Written by Aaryaman Sagar for EECS 281 @ The University of
                 Michigan - Ann Arbor}
   \pagestyle{empty}
}
\thispagestyle{firststyle}

This is the first lab in a series of 2 labs which will end with you writing a
custom untyped memory allocator very much like the C \texttt{malloc()} function!

\section{Transparent double ended linked list}
Your first task in this lab will be to write a doubly ended linked list that
does not manage memory.  Such a list is often times useful when the underlying
memory structure for the system is not known.  For example this can be useful
in embedded systems where there is no operating system or when writing
programs right on top of the kernel memory interface (take EECS 482 for more
details).

The doubly ended linked list does not contain any bookkeeping other than the
head and tail pointers, no size variable is included.  Each node is passed to
the linked list with a pointer and the linked list changes its connections and
adds it to the linked list.

\subsection{Deliverables}

\begin{lstlisting}
    void push_back(TransparentNode*)
\end{lstlisting}
This function will accept a node type by pointer and put that pointer at the
back of the linked list.  All invariants must be satisfied after calling this
function.  For example, the tail pointer must equal to the pointer passed in
to this function when the function call finishes, also the head pointer must
be updated if there were no elements in the linked list prior to this function
call.

\begin{lstlisting}
   void push_front(TransparentNode*)
\end{lstlisting}
This function very much like \texttt{push\_back()} accepts a node by pointer
and puts that node into the linked list at the very beginning.  The head
pointer must always be updated and all invariants in the linked list must be
satisfied after this call finishes.

\begin{lstlisting}
    NodeIterator insert(NodeIterator, TransparentNode*)
\end{lstlisting}
The insert function accepts an iterator as an argument along with a node
pointer that it must insert right before the element pointed to by the
iterator.  The function must take into account the fact that the iterator
passed as an argument might be past the end of the linked list, in which case
the result will be the same as calling \texttt{push\_back()} on an empty
linked list.  The function must return an iterator to the inserted element
back to the user.

\begin{lstlisting}
    NodeIterator erase(NodeIterator)
\end{lstlisting}
This method will be passed an iterator and it should remove the element
pointed to by the iterator, and update head and tail pointers accordingly.
After deletion, this should return an iterator following the removed element.

\bigskip
\subsection{Misc}

This linked list implementation does not manage any memory and hence is called
transparent.  As such you are not supposed to call either new or delete for
any code that you write for this lab.

A lot of the linked list functionality is already implemented for you.
Including the iterator class that you will have to use, so be sure to take a
look in the implementation whenever you have questions.

This library follows the convention of splitting the interface and
implementation files into two seperate files, one ending with an \texttt{.ipp}
extension and one ending with a \texttt{.hpp} extension.  You are only
supposed to write code in the \texttt{.ipp} file in the code segments marked
with a \texttt{TODO} symbol.

\newpage
\section{Submitting your work}
Please submit the implementation (\texttt{TransparentList.ipp}) file in a
tarball to the autograder, you can use the following command to generate the
tarball.
\begin{lstlisting}
    tar -czvf lab08.tar.gz TransparentList.ipp
\end{lstlisting}

\end{document}


